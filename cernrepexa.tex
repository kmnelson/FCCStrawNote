\documentclass{cernrep}
\usepackage{doi}
\usepackage[
backend=biber,
style=numeric,
sorting=none
]{biblatex}
\usepackage{hyperref}
\hypersetup{
    colorlinks = true,
    urlcolor = cyan,
    citecolor = black,
}

\addbibresource{bibliography.bib}

\begin{document}


\title{Simulation and performance studies of a straw tube tracker detector concept for the FCC-ee}
\author{M. Arias, Y. Li, K. Nelson, J. Qian}
\institute{University of Michigan}


\begin{abstract}
The planed Future Circular Collider will operate first as a lepton collider in an $e^+e^-$ mode, the FCC-ee.
In contrast to the Large Hadron Collider, the current energy-frontier machine, the FCC-ee will have precise control on the center-of-mass energy to below the 0.2\% level.
Therefore, a high-precision inner tracking detector with transverse momentum resolution on the order of 0.2\% is required to fully take advantage of the clean collider environment.
Existing detector concepts to meet these needs include the IDEA drift chamber and the CLD silicon tracker.
However, we study a third option: a thin-wall straw tube tracker.
Such a tracker could offer lower material budget than a full silicon detector and improved particle identification compared to a drift chamber.
In this note we briefly present an overview of the detector concept and its possible advantages and flexibility before presenting a detailed accounting of simulation studies and geometric optimization of the detector layout.
\end{abstract}

\keywords{FCC-ee; straw tracker; detector simulation}

\maketitle

\section{Introduction}



The Future Circular Collider (FCC) \cite{FCCeeconcept,fccatcern} is a planned collider experiment which will have a lepton colliding stage, FCC-ee, at center-of-mass energies to cover the $Z$ pole as well as $WW$, $ZH$ and $t\bar{t}$ production.
Existing detector concepts include ALLEGRO \cite{allegroconcept}, IDEA \cite{ideaconcept,idea2}, and CLD \cite{cldconcept}.



\printbibliography

\end{document}
